

{\bfseries my\-\_\-package} is ... In addition to providing an overview of your package, this is the section where the specification and design/architecture should be detailed. While the original specification may be done on the wiki, it should be transferred here once your package starts to take shape. You can then link to this documentation page from the Wiki.\section{Code A\-P\-I}\label{index_codeapi}
Provide links to specific auto-\/generated A\-P\-I documentation within your package that is of particular interest to a reader. Doxygen will document pretty much every part of your code, so do your best here to point the reader to the actual A\-P\-I.

If your codebase is fairly large or has different sets of A\-P\-Is, you should use the doxygen 'group' tag to keep these A\-P\-Is together. For example, the roscpp documentation has 'libros' and 'botherder' groups so that those can be viewed separately. The rospy documentation similarly has a 'client-\/api' group that pulls together A\-P\-Is for a Client A\-P\-I page.\section{R\-O\-S A\-P\-I}\label{index_rosapi}
Every R\-O\-S name in your code must be documented. Names are very important in R\-O\-S because they are the A\-P\-I to nodes and services. They are also capable of being remapped on the command-\/line, so it is V\-E\-R\-Y I\-M\-P\-O\-R\-T\-A\-N\-T T\-H\-A\-T Y\-O\-U L\-I\-S\-T N\-A\-M\-E\-S A\-S T\-H\-E\-Y A\-P\-P\-E\-A\-R I\-N T\-H\-E C\-O\-D\-E. It is also important that you write your code so that the names can be easily remapped.

List of nodes\-:
\begin{DoxyItemize}
\item {\bfseries node\-\_\-name1} 
\item {\bfseries node\-\_\-name2} 
\end{DoxyItemize}



\subsection{node\-\_\-name}\label{index_node_name}
node\-\_\-name does (provide a basic description of your node)\subsubsection{Usage}\label{index_Usage}
\begin{DoxyVerb}$ node_type1 [standard ROS args]
\end{DoxyVerb}


\begin{DoxyParagraph}{Example}

\end{DoxyParagraph}
\begin{DoxyVerb}$ node_type1
\end{DoxyVerb}
\subsubsection{R\-O\-S topics}\label{index_topics}
Subscribes to\-:
\begin{DoxyItemize}
\item {\bfseries \char`\"{}in\char`\"{}}\-: [std\-\_\-msgs/\-Foo\-Type] description of in
\end{DoxyItemize}

Publishes to\-:
\begin{DoxyItemize}
\item {\bfseries \char`\"{}out\char`\"{}}\-: [std\-\_\-msgs/\-Foo\-Type] description of out
\end{DoxyItemize}\subsubsection{R\-O\-S parameters}\label{index_parameters}
Reads the following parameters from the parameter server


\begin{DoxyItemize}
\item {\bfseries \char`\"{}$\sim$param\-\_\-name\char`\"{}} \-: {\bfseries }[type] description of param\-\_\-name
\item {\bfseries \char`\"{}$\sim$my\-\_\-param\char`\"{}} \-: {\bfseries }[string] description of my\-\_\-param
\end{DoxyItemize}

Sets the following parameters on the parameter server


\begin{DoxyItemize}
\item {\bfseries \char`\"{}$\sim$param\-\_\-name\char`\"{}} \-: {\bfseries }[type] description of param\-\_\-name
\end{DoxyItemize}\subsubsection{R\-O\-S services}\label{index_services}

\begin{DoxyItemize}
\item {\bfseries \char`\"{}foo\-\_\-service\char`\"{}}\-: [std\-\_\-srvs/\-Foo\-Type] description of foo\-\_\-service
\end{DoxyItemize}\section{Command-\/line tools}\label{index_commandline}
This section is a catch-\/all for any additional tools that your package provides or uses that may be of use to the reader. For example\-:


\begin{DoxyItemize}
\item tools/scripts (e.\-g. rospack, roscd)
\item roslaunch .launch files
\item xmlparam files
\end{DoxyItemize}\subsection{script\-\_\-name}\label{index_script_name}
Description of what this script/file does.\subsubsection{Usage}\label{index_Usage}
\begin{DoxyVerb}$ ./script_name [args]
\end{DoxyVerb}


\begin{DoxyParagraph}{Example}

\end{DoxyParagraph}
\begin{DoxyVerb}$ ./script_name foo bar
\end{DoxyVerb}
 